\documentclass[12pt]{article}
\usepackage{amsmath}
\usepackage{pdflscape}
\usepackage{graphicx}
\usepackage{booktabs}
\usepackage{multirow}

\title{Word of Mouth Learning in Mormon Congregations}
\author{Graton Gathright \\and\\Christopher D. Wignall}

\begin{document}
\newcommand{\se}[1]{\footnotesize{(#1)}}
\newcommand{\capbot}[1]{\vspace{8pt}\footnotesize\raggedright #1}
\pagestyle{empty}

\maketitle
%%%%%%%%%%%%%%%%%%%%%%%%%%%%%%%%%%%%%%%%%%%%%
%VIM stuff
%Word Count - g<ctrl+g>
%zg adds word to ``good word list''
%<F8> enters spellcheck mode

%%%%%%%%%%%%%%%%%%%%%%%%%%%%%%%%%%%%%%%
%Writing stuff
%real estate agent, not realtor
%Ward not congregation
%homeowner
%
%From Cochrane:
%1. Describe the economic mechanism that caused the dispersion in the RHS variables
%2. Describe the economic mechanism that constitutes the error term
%3. Why is the error term uncorrelated with the RHS?
%6. Describe the source of variation in the data that drives the estimates for every single number presented (across ward? within listing?).
%7. Whose behavior is being modeled?
%8. Are you sure about the direction of causality?
%9. Which controls should be in the regression? Which should not?
%
%From colleagues, books:
%More ``Mormons''
%Outline Preview
%Zillow hole

%%%%%%%%%%%%%%%%%%%%%%%%%%%%%%%%%%%%%%%%%%%%%%%
%Research stuff to do and think about
%Different Logit specifications with robust standard errors
%Think of alternative hypotheses that need to be ruled out
%Why include realtor characteristics?
%Stories
%Better check of similarity in, out of ward
%Other NBHD definitions
%Realtor Geographic specialization (CDF of #wards?)
%Interpretatione
%Supply side issues
%Mismeasurement of wards
%Outliers
%Count observations
%Days on Market
%Park City vs. Orem
%Borders
%Rural vs. Urban
%Overlapping in time
%Selling Agent
%Offices
%Why drop the first 5 years?
%Storiesmidrule
{\abstract
    We analyze social learning by members of the Church of Jesus Christ of Latter-day Saints (Mormons) as they select
    real estate agents to list their homes. The Mormon congregation is useful for social analysis: it is important to its members and assigned
    geographically. We use the geographic relationship between homes to control for selection into wards and show that the average home seller is much
    more likely to choose the same real estate agent as his neighbor if they are assigned to the same congregation. We also present evidence that some
    of a neighbor's influence is due to word-of-mouth communication.
}

\section*{Introduction}
    A person choosing among options with unknown values can improve his outcome by gathering information before making his decision.
    If other people have previously chosen from the same set of options, he has an opportunity to learn from them.

    Social learning is an example of a ``peer effect.'' There are many papers that investigate how individuals are influenced by their peers, but 
    attempts to measure peer effects are complicated by a difficult identification problem: unobserved characteristics that influence behavior may
    also influence which relationships form. If some omitted variable leads two people to make similar decisions while also increasing the probability
    that they become peers, then estimates of the social effect will be biased upward. For example, if people with tastes for risky behavior are more
    likely to smoke and also tend to be friends with other risk lovers, then estimates of peer influence on smoking will exhibit a positive omitted
    variable bias.

    We quantify social learning in congregations of The Church of Jesus Christ of Latter-day Saints (Mormon). A Mormon congregation, called a ward, is
    an important social network for its members. The ward is defined by a set of geographic boundaries, and so a Mormon is assigned to a ward based on
    street address in much the same way that a child is assigned to a public elementary school. By design, ward boundaries seldom lie coincident
    with significant neighborhood boundaries (like major roads or subdivision boundaries). Because of this, once we control for the geographic
    relationship between two homes, we can treat whether the residents are in the same ward as the result of a random assignment.

    We examine how home sellers are influenced by their neighbors when they select a real estate agent. We find that the average seller is more
    likely to choose the same real estate agent as a neighbor if both are assigned to the same ward. In addition, we present evidence that individuals
    are responding directly to the private information of their neighbors, suggesting that at least some of the communication is direct
    (word-of-mouth) rather than observational (where individuals observe neighbors' behavior and make inference about private information).

    Researchers have employed a variety approaches to estimate social effects in the presence of omitted variables. For example, Duflo and Saez (2002)
    evaluate peer effects on retirement saving decisions by employees of a large university treatingmembers of the same academic department as peers.
    They use an instrument to correct for the omitted variable bias and find a significant social effect. In a similar paper, Sorensen (2006) looks at
    health plan choice by employees of the University of California system. He uses the panel structure of his data to allow for unobserved
    heterogeneity between departments. He finds a significant social effect, especially by peers that are similar in age or family structure. 

    We make three main contributions in this paper. First, we employ a novel social network that, conditional on geography, does not suffer from
    an unobserved selection process. In practice, this means that our identification of social effects does not rely on the assumptions for the
    instrumental variables or fixed effects estimators. Second, we quantify social learning in a specific economic environment: real estate brokerage.
    Finally, we offer evidence that some of this learning occurs through direct (word-of-mouth) communication. This has important
    welfare implications since pure observational learning faces a higher probability of an information cascade and inefficient
    herding\footnote{For an overview of the literature on information cascades, see Bikhchandani et al (1998).}.
    Furthermore, since the relationship between home seller and real estate agent is not likely to be repeated, a divergence in their preferences can
    lead to inefficient outcomes. Social learning presents an opportunity for the agent to maintain a reputation and direct communication between
    consumers can offer the agent better incentives to please the current client in
    order to improve future business.

\section*{Identification}
    Our objective is to evaluate how individuals are influenced by their peers. Our main prediction is that, if a home seller learns about real estate
    agents from her ward social network, then she is more likely to respond to the real estate agent choice of a neighbor if they are assigned to the
    same ward. To see this, consider the two types of social learning described above. First, if learning is purely observational, then home sellers
    see which real estate agent their peers choose and infer that the peer has private information that the selected real estate agent is of high
    quality, which increases the sellers' expected payoffs from choosing that agent.  

    If, on the other hand, there is direct communication between peers, then the home seller can get information on the experience her friend had with
    his selected agent. If the friend had a good experience, her beliefs about the agent's quality increase and she is more likely to choose that
    agent. If the friend had a bad experience, the seller's beliefs about quality decrease and she is less likely to choose that agent. If information
    is more likely to be positive than negative, then, on average, home sellers are more likely to choose the same real estate agent as their peers.
    If the information is more likely to be negative, then the opposite is true. 

    Both types of learning imply that home sellers are more likely to respond to the choice of a neighbor if they are peers. Since people in
    the same ward are more likely to be peers, ceteris paribus, we predict that the probability that a home seller chooses the same real estate agent
    as her neighbor is sensitive to whether they are in the same ward.
    
    In many settings, the simultaneity of peer influence complicates the estimation of social effects: Alices behavior incluences Bob's behavior
    which, in turn, influences Alice. To identify the direct influence of one person on another in our setting, it is natural to assume that influence
    between a pair of peers goes only one way.  This is the case if, for example, the friend who chooses first influences the friend who chooses
    second, but the friend who chooses second does not simultaneously influence the first.

    The more fundamental identification problem in the estimation of peer effects arises from the omission of the relevant variables that determine
    the existing social network - the ``reflection problem\footnote{For a full discussion of the reflection problem in social effects, see Manski
    (1993).}.''
    The typical economic study is interested in how peer behavior impacts individual choices. This ``endogenous effect'' has an important multiplier
    property: a change in one person's behavior can change the behavior of his peers. His peers behavior then affects his behavior again.
    Quantifying this peer effect, then, is an important part of choosing optimal policy or marketing. At the same time, peers tend to share similar
    characteristics that may affect their behavior. For example, if people tend to form relationships with people with similar levels of intelligence
    and intelligence influences years of education, then there will be a correlation in years of education among peers.  We will not be able to
    identify the endogenous effect from the ``correlated effects'' arising from selection into relationships if these variables (like intelligence in
    the example above) are not observed by the econometrician. Estimates of the endogenous effect with have an upward omitted variable bias. 

    Our identifying assumption is that, once we condition on the geographic relationship between two neighbors, the ward relationship between them is
    random.

    Wards are defined by geographic boundaries designed to incorporate 300-500 members. In order to participate fully in their faith, members must
    attend the ward to which they are assigned. In Mormon theology, there are two levels of church worship. The first and most basic is the weekly
    Sunday service, held in local chapels and open to the public. Each ward has its own set of meetings that are managed by the ward members and
    leaders. The second type of worship occurs in Mormon temples, participation in which is limited to members that are in good standing and approved
    by their ward leaders. One of the requirements for good standing is regular participation in the Sunday services of the ward to which they are
    assigned. It would be difficult to overstate the importance of temple worship in Mormon theology. Because of this, practicing Mormons almost
    always attend their assigned wards; there are very few exceptions.

    In this paper, we focus our attention on Utah County, where about 89\% of the population is
    Mormon\footnote{From the \emph{Religious Congregations and Membership in the United States, 2000}, collected by Association of Statisticians of
    American Religious Bodies.}.

    Home characteristics do not change continuously over space: there are abrupt changes in house (and resident) characteristics at subdivision
    boundaries and geographic features like rivers and parks. If ward boundaries systematically coincide with these discontinuous changes, then our
    identifying assumption will be violated.

    By construction, however, ward boundaries tend to not lie on neighborhood boundaries. In private conversation, church leaders involved in setting
    ward boundaries in Utah County have explained that one of the primary objectives is to ensure homogeneity across wards: they want neighboring
    wards to have very similar demographic compositions. In practice, this means that ward boundaries are likely to cut across neighborhood
    boundaries, not lie coincident with them.

    Regular participation in the ward involves frequent interpersonal interaction with co-congregants. Since the Mormon Church has no paid clergy at
    the ward level, all tasks are performed by individual lay members. For example, all adults are assigned a list of families that they are expected
    to visit on at least a monthly basis. According to a 2008 Survey by the Pew Research Center\footnote{\emph{US Religious Landscape Survey}, Pew
    Forum on Religion and Public Life}, 75\% of Mormons attend religious services at least once a week and 92\% of them are formal members of their
    congregations (wards). In addition, 77\% participate in non-worship activities at church with 63\% participating in social activities at church at
    least monthly.
    
\section*{Data}
    Our home transactions and characteristics data come from the Wasatch Front Regional Multiple Listing Service, the largest MLS in Utah and the only
    MLS active in Utah County during our sample period. We have information on listings for single family residences in Utah County from 1997-2006.

    The contract between a home seller and her real estate agent is called a listing agreement. The real estate agent agrees to market the home in
    exchange for a payment that is typically expressed as a percentage of the sales price due at closing. The MLS requires that its member agents post
    their new listings to the MLS within 72 hours of signing a listing agreement. Most real estate agents are members of the MLS and so we have a
    record of nearly every home listed by an agent in Utah County during the sample period.

    The MLS listing record contains home characteristics (square footage, number of bedrooms, street address, etc.) and identifying information for
    agents involved in the transaction. Also included are the listed price, whether the property sold during the listing period, and, if sold, the
    sales price.

    Using the Google Maps application programming interface (API), we locate each home on a map. Because of the large number of listings, it is not
    computationally feasible to evaluate the relationships between every pair. We limit our attention to listings that are located within one quarter
    mile and listed within five years of each other. For each pair, we calculate the geographic distance between them (as the crow flies). 
    
    The Utah County Assessor's Office has divided the county into neighborhoods based on development stage: contiguous homes that were developed at
    the same time belong to the same neighborhood. Using map data from the county, we determine whether each pair of homes is assigned to the same
    county-defined neighborhood.  Similary, using data from the U.S. Census Bureau (TigerLine), we determine whether neighbors are belong to the same
    census block.

    A person's home address determines the ward to which he is assigned. The Mormon Church provides a ward lookup tool on its website (``Meetinghouse
    Locator,'' www.lds.org). We queried the website with the street address of each of our listings in order to determine its assigned ward.

    The church leaders responsible for setting the ward boundaries prefer to create homogeneity between wards. Our data allow us to evaluate how
    successful they are. If ward boundaries systematically coincide with the discontinuous changes in neighborhoods, then homes should be, on average,
    more similar to other homes in the same ward. For each pair of homes (that are within one quarter of a mile and listed within five years of each
    other), we calculate the difference in observed characteristics. We then check whether, once we control for geography, the ward relationship
    explains some of the variation in the differences in characteristics by regressing them on the geographic and ward relationship between
    neighbors. So, if homes $i$ and $j$ are neighbors, then:
    \[ y_{i,j}  = \alpha + \beta W_{i,j} + \gamma_1 D_{i,j} + \gamma_2 D^2_{i,j} + \delta N_{i,j} + \varepsilon_{i,j} \]
    where $ y_{i,j} $ is the absolute difference in an observed characteristic between homes $ i $ and $ j $, (for example $ |ln(\textrm{square
    feet}_i) - ln(\textrm{square feet}_j)| $). $ W_{i,j} $ is equal to 1 if homes $i$ and $j$ are in the ame ward, $ D_{i,j} $ is the geographic
    distance between homes $i$ and $j$, and $ N_{i,j} $ equals 1 if homes $i$ and $j$ are assigned to the same county-defined neighborhood.
    If homes are, on average, more similar to other homes in the same ward, then the estimate of $ \beta $ should have a negative coefficent. We
    perform this regression on a number of observed home characteristics. The estimates of $\beta$ from each of these regressions are presented in
    the Ward column of Table 4. We find not evidence that homes tend to be systematically more similar to other homes assigned to the same ward once
    we condition on geography.

    As a comparison to our ward social network definition, we also consider whether census blocks boundaries tend to coincide with neighborhood
    boundaries. Column 2 of Table 4 shows that, even conditional on geography, homes in the same census block tend to be more similar to each other
    then to other neighboring homes: for example, they are approximately 4\% more similar in square footage and about 50 years more similar in age
    than homes the same distance away but in different census blocks. This is especially pertinent since some researchers estimate social effects
    using the census block as the relevant peer group (see, for example, Bayer et al (2005)). 

\section*{Results}
    To evaluate peer influence, we estimate the probability that a home seller chooses the same real estate agent as her neighbor. If ward members
    learn from each other, then the probability that neighbors choose the same real estate agent should be different if they are assigned to the same
    ward. We condition on the geographic relationship between the neighbors using the distance between them and whether they belong to the same
    county-defined neighborhood. We measure distance in quarter miles so that it lies in the zero to one interval (since we only consider pairs of
    homes that are within a quarter mile and listed within five years of each other). We also include distance squared.

    We present in Table 5 the coefficient estimates from a linear probability model:
    \[P(i \textrm{and} j \textrm{choose the same agent}) 
            = \alpha + \beta W_{i,j} + \gamma_1 D_{i,j} + \gamma_2 D^2_{i,j} + \delta N_{i,j} + \varepsilon_{i,j}\]
    Table 6 shows the results from a logit regression. The predicted
    probabilities from the two specifications are very similar. We focus on the results from the linear probability model for ease of exposition.

    We calculate multi-way cluster robust standard errors using the method developed by Cameron, Gelbach, and Miller (2007) for both specifications.
    Our observations are on pairs of listings. Since each listing in the pair belongs to many different pairs, we estimate standard errors that are
    robust to clustering on the listing for both listings in the pair.

    The results from the linear probability model, presented in Table 5, suggest that a home seller is substantially more likely to choose the same
    real estate agent as her neighbor if they are assigned to the same ward. For an intermediate distance (one eighth of a mile, distance = .5), the
    probability they choose the same agent is 1.3\% if the are assigned to different wards and 2.4\% if they are assigned to the same ward.
    
    The distance between two neighbors is an important determinant of the probability that they choose the same real estate agent. While some of this
    effect may be due to social influence (near neighbors are more likely to be in the same peer group), it may also represent small scale geographic
    specialization and marketing by the agents.

    The importance of the ward social network can be described in terms of the change in geography that will offset a ward relationship. For example,
    a neighbor that is assigned to a different ward and is a sixteenth of a mile away is as influential as a neighbor that is 50\% farther away and is in the
    same ward. A non-ward neighbor that is a tenth of a mile away is as influential as a ward neighbor that is 75\% farther away.

    We compare these results to the social effects from a social network defined by census block. Column 2 of Table 5 shows the results. The estimated
    influence of peers in the same census block is about 40\% greater than that estimated by the ward peer group. We find very similar effects for
    geographic distance under both definitions of group. Census blocks, however, tend to be defined by the characteristics of the homes and their
    occupants. The estimate of social effect for census blocks does not identify the endogenous effect from the correlated effect.

    If a social group has subgroups, then people should be more influenced by peers in their subgroup. We do not have data on any within ward
    subgroups, but we can evaluate how peer influence is affected by characteristics of the neighbors that would decrease the strength of their social
    ties. First, we analyze how the geographic relationship between neighbors impacts peer influence. We do not know any characteristics of the home
    sellers except the observed characteristics of their houses. We use the difference in house characteristics as proxies for difference in
    individuals. People may be more influenced by peers with a similar level of wealth or with the same family structure. Since house price is highly
    correlated with wealth, we look at how differences in the list price between the two homes affect peer influence. Similarly, since people with
    children are more likely to have more bedrooms, we look at how the difference in the number of bedrooms impacts peer influence. We also consider
    how geographic distance weakens social effects. The results are shown in
    column 3 of table 5.

    We find that these separating factors do attenuate peer influence. The difference in the number of bedrooms and list price have little effect for
    neighbors that are assigned to different wards. The do have a significant impact on peer influence for ward neighbors. A 1\% difference in the
    list price of neighbors in the same ward corresponds to a decrease in the predicted probability of choosing the same real estate agent of three
    percentage points. A difference of one bedroom corresponds to a decrease of 1.6 percentage points. Additionally, geographic distance substantially
    decreases peer influence of ward members.

    We have offered evidence of social influence by people assigned to the same ward. We have not addressed the source of this influence. If home
    sellers are learning from their peers by observing choices and making inference about private information, then the choices of identical peers
    will have identical influence. If, on the other hand, peers are communicating directly, then they may respond to their neighbors' private
    information, including information about outcomes.

    Utah is a ``non-disclosure'' state: the county governments do not record the transaction prices of real estate. The only repository of home sale
    prices in Utah County is the WFRMLS, which is not available to the public. This means that, for the average person, the only source of information
    about sales price is from one of the individuals involved in the transaction; it is private information. 
    
    Home sellers prefer a higher sales price, ceteris paribus. If they are learning from their peers, they are more likely to choose the same real
    estate agent as a neighbor if that agent sold the neighbor's house for a relatively high price and less likely to choose him if he sold the house
    for a relatively low price. For each transaction, we calculate the percent difference between sales price and list price. We categorize
    transactions in the top decile of the distribution of percent difference ``very high price'' and transactions in the bottom decile ``very low
    price.'' We then analyze how these outcomes affect peer influence. The results are in column 4 of table 5.

    We find effects for peers with intermediate outcomes that are similar to the effects in the baseline specification. In addition, for neighbors in
    the same ward we find a large premium associated with a very high price and a large penalty for a very low price. The additional influence of ward
    members with very high prices is double that of those with intermediate outcomes and the penalty from a very low price almost cancels the ward
    effect completely.


\begin{table}[ht]
    \centering
    \caption{Summary Statistics of House Characteristics}
    \begin{tabular}{@{}lccc@{}} \toprule
        Variable        & Mean      & Standard Deviation    & Median \\ \midrule
        List Price      & 214970    & 125474                & 174000 \\
        Square Feet     & 2659      & 1256                  & 1256 \\
        Acres           & 0.35      & 1.1                   & 0.22 \\
        Bedrooms        & 3.9       & 1.2                   & 4 \\
        Bathrooms       & 2.5       & 1.0                   & 2 \\
        Garage Capacity & 1.5       & 0.81                  & 2 \\
        Patios          & 0.46      & 1.1                   & 0 \\
        Decks           & 0.28      & 0.44                  & 0 \\
        Wet Bars        & 0.40      & 0.52                  & 0 \\
        Fire Places     & 0.64      & 0.76                  & 0 \\
        Sold Indicator  & 0.59      & 0.49                  & 1 \\
        Sold Price      & 191828    & 99761                 & 162000 \\ \bottomrule
    \end{tabular}
\end{table}
        

\begin{table}[ht]
    \centering
    \caption{Summary Statistics of neighbors and relationships}
    \begin{tabular}{@{}p{5cm}ccc@{}} \toprule
        Variable                                                                        & Mean  & Standard Deviation    & Median \\ \midrule
        Number of Neighbors                                                         & 60 & 47                    & 52 \\ \addlinespace
        Number of Neighbors in the Same Ward                                        & \multirow{2}{*}{28} & \multirow{2}{*}{25} & \multirow{2}{*}{22} \\ \addlinespace
        Percent Neighbors in the Same Ward                                          & \multirow{2}{*}{50} & \multirow{2}{*}{25} & \multirow{2}{*}{46} \\ \addlinespace
        Number of Neighbors in the Same Neighborhood                                & \multirow{2}{*}{39} & \multirow{2}{*}{39} & \multirow{2}{*}{29} \\ \addlinespace
        Percent Neighbors in the Same Neighborhood                                  & \multirow{2}{*}{62} & \multirow{2}{*}{31} & \multirow{2}{*}{71} \\ \addlinespace
        Number of Neighbors in the Same Census Block                                & \multirow{2}{*}{15} & \multirow{2}{*}{29} & \multirow{2}{*}{6} \\ \addlinespace
        Percent Neighbors in the Same Census Block                                  & \multirow{2}{*}{22} & \multirow{2}{*}{25} & \multirow{2}{*}{13} \\ \addlinespace
        Percent of Neighbors in the Ward that Chose the Same Real Estate Agent      & \multirow{3}{*}{5.0} & \multirow{3}{*}{15} & \multirow{3}{*}{0} \\ \addlinespace
        Percent of Neighbors Outside the Ward that Chose the Same Real Estate Agent & \multirow{3}{*}{1.5} & \multirow{3}{*}{16} & \multirow{3}{*}{0} \\ \bottomrule
    \end{tabular}
\end{table}

\begin{table}[ht]
    \centering
    \caption{Summary Statistics of Wards and Agents}
    \begin{tabular}{@{}lccc@{}} \toprule
        Variable                    & Mean  & Standard Deviation    & Median \\ \midrule \addlinespace
        Ward (N = 832) \\ \cmidrule(r{2cm}){1-1}
        Number of Listings          & 72    & 42                    & 63 \\
        Number of Agents            & 51    & 26                    & 48 \\ \addlinespace
        Agent (N = 5904) \\ \cmidrule(r{2cm}){1-1}
        Number of Listings          & 10    & 31                    & 2 \\
        Number of Wards             & 7     & 19                    & 2 \\ \bottomrule
    \end{tabular}
\end{table}
    
\begin{table}[ht]
    \caption{Predicting variation in observed house characteristics}
    \begin{center}
    \resizebox{12cm}{!} {
    \begin{tabular}{@{}lcc@{}}\toprule
        \multirow{2}{*}{Variable}               & Ward  & Census Block  \\ & \se{Standard Error}   & \se{Standard Error} \\
        \midrule
        \multirow{2}{*}{Log List Price}         & 0.004 & 0.003         \\ & \se{0.0003}           & \se{0.0003} \\ 
        \addlinespace 
        \multirow{2}{*}{Log Square Footage}     & -0.007& -0.04         \\ & \se{0.004}            & \se{0.004}\\ 
        \addlinespace 
        \multirow{2}{*}{Log Acres}              & 0.005 & 0.02          \\ & \se{0.001}            & \se{0.001}\\ 
        \addlinespace 
        \multirow{2}{*}{Number of Bedrooms}     & -0.02 & -0.09         \\ & \se{0.001}            & \se{0.001}\\ 
        \addlinespace 
        \multirow{2}{*}{Number of Bathrooms}    & -0.01 & -0.03         \\ & \se{0.001}            & \se{0.001}\\ 
        \addlinespace 
        \multirow{2}{*}{Garage Capacity}        & -0.02 & -0.22         \\ & \se{0.001}            & \se{0.001}\\ 
        \addlinespace 
        \multirow{2}{*}{Patios}                 & -0.01 & -0.01         \\ & \se{0.001}            & \se{0.001}\\ 
        \addlinespace 
        \multirow{2}{*}{Decks}                  & -0.004& -0.02         \\ & \se{0.001}            & \se{0.001}\\ 
        \addlinespace 
        \multirow{2}{*}{Wet Bars}               & -0.002& -.03          \\ & \se{0.001}            & \se{0.001}\\
        \addlinespace 
        \multirow{2}{*}{Fire Places}            & -0.01 & -0.12         \\ & \se{0.001}            & \se{0.001}\\ 
        \addlinespace 
        \multirow{2}{*}{Year Built}             & 2.2   & -51.1         \\ & \se{0.36}             & \se{0.38} \\ 
        \bottomrule
    \end{tabular}
    }
    \end{center}
    \capbot{
        Each row presents the estimates for $\beta$ in the regression \[y_{ij} = \alpha + \beta W_{ij} + \gamma_1 D_{ij} + \gamma_2 D_{ij} +
        \delta N_{ij} + \varepsilon_{ij}\] where $y_{ij}$ is the absolute difference in the home characteristic listed in the first column, $W_{ij} =
        1$ if the homes are in the same ward, $D_{ij}$ is the geographic distance between the homes, and $N_{ij} = 1$ if the homes are in the same
        county-defined neighborhood. The third column presents the estimate of $\beta$ using the census block instead of the ward.
    }
\end{table}

\begin{table}[ht]
  \caption{Results from Linear Regression: Probability of choosing the same real estate agent}
  \begin{center}
  \resizebox{12cm}{!} {
  \begin{tabular}{@{}lcccc@{}}\toprule
      \multirow{2}{*}{Chose the Same Real Estate Agent}   & (1)     & (2)     & (3)     & (4)    \\& Wards  & Census Blocks & Weak Ties & Learning \\ 
      \midrule
      \multirow{2}{*}{In the Same Ward}                   & 1.0**   &         & 6.4**   & 0.96** \\& \se{0.05} &           & \se{0.05} & \se{0.05} \\ 
      \multirow{2}{*}{In the Same Census Block}           &         & 1.4**   &         &        \\&           & \se{0.07} &           &           \\ 
      \multirow{2}{*}{Distance (in Quarter Miles)}        & -15.9** & -15.1** & -5.1**  & -15.8* \\& \se{0.50} & \se{0.49} & \se{0.49} & \se{0.49} \\ 
      \multirow{2}{*}{Distance Squared}                   & 10.5**  & 9.7**   & 2.9**   & 10.4** \\& \se{0.36} & \se{0.36} & \se{0.36} & \se{0.36} \\ 
      \multirow{2}{*}{In the Same Neighborhood}           & 0.3**   & 0.3**   & 0.2**   & 0.3**  \\& \se{0.04} & \se{0.04} & \se{0.04} & \se{0.04} \\ 
      \multirow{2}{*}{Distance * Same Ward}               &         &         & -12.2** &        \\&           &           & \se{0.83} &           \\ 
      \multirow{2}{*}{Distance Squared * Same Ward}       &         &         & 8.6**   &        \\&           &           & \se{0.63} &           \\ 
      \multirow{2}{*}{Same Neighborhood * Same Ward}      &         &         & -0.2    &        \\&           &           & \se{0.10} &           \\ 
      \multirow{2}{*}{Absolute Difference in Square Feet} &         &         & -0.1*   &        \\&           &           & \se{0.05} &           \\ 
      \multirow{2}{*}{Absolute Difference in Bedrooms}    &         &         & -0.5**  &        \\&           &           & \se{0.07} &           \\ 
      \multirow{2}{*}{Absolute Difference in List Price}  &         &         & -0.7**  &        \\&           &           & \se{0.07} &           \\ 
      \multirow{2}{*}{Difference in Square Feet * Ward}   &         &         & -0.3*   &        \\&           &           & \se{0.18} &           \\ 
      \multirow{2}{*}{Difference in Bedrooms * Ward}      &         &         & -1.7**  &        \\&           &           & \se{0.15} &           \\ 
      \multirow{2}{*}{Difference in List Price * Ward}    &         &         & -3.3**  &        \\&           &           & \se{0.17} &           \\ 
      \multirow{2}{*}{Very High Price}                    &         &         &         & 0.2**  \\&           &           &           & \se{0.08} \\ 
      \multirow{2}{*}{Very Low Price}                     &         &         &         & 0.2    \\&           &           &           & \se{0.10} \\ 
      \multirow{2}{*}{Very High Price * Ward}             &         &         &         & 1.2**  \\&           &           &           & \se{0.20} \\ 
      \multirow{2}{*}{Very Low Price * Ward}              &         &         &         & -0.9** \\&           &           &           & \se{0.14} \\ 
      \multirow{2}{*}{Constant}                           & 6.4**   & 6.4**   & 3.3**   & 6.4**  \\& \se{0.17} & \se{0.17} & \se{0.23} & \se{0.17} \\
      \midrule
      N                                                   & 3029739 & 3029737 & 3024191 & 3029737\\
      Adjusted $R^2$                                      & 0.01    & 0.02    & 0.02    & 0.01 \\
      \bottomrule
    \end{tabular}
    }
    \end{center}
    \capbot{Linear Probability Model: 
            \[P(i \textrm{and} j \textrm{choose the same agent}) 
                = \alpha + \beta W_{i,j} + \gamma_1 D_{i,j} + \gamma_2 D^2_{i,j} + \delta N_{i,j} + \varepsilon_{i,j}\]
            Standard errors are robust to clustering at the listing level in both direction (both listings in a pair).
            }
        
\end{table}

\begin{table}[ht]
  \centering
  \caption{Results from Logit Regression: Probability of choosing the same real estate agent}
  \resizebox{12cm}{!} {
  \begin{tabular}{@{}lcccc@{}}\toprule
    \multirow{2}{*}{Chose the Same Real Estate Agent}   & (1)     & (2)     & (3)     & (4)    \\& Wards & Census Blocks & Weak Ties & Learning \\ 
    \midrule
    \multirow{2}{*}{In the Same Ward}                   & 0.60**  &         & 1.42**  & 0.60** \\& \se{0.03} &           & \se{0.13} & \se{0.03} \\ 
    \multirow{2}{*}{In the Same Census Block}           &         & 0.61**  &         &        \\&           & \se{0.03} &           &           \\ 
    \multirow{2}{*}{Distance (in Quarter Miles)}        & -4.21** & -3.78** & -2.94** & -4.19**\\& \se{0.13} & \se{0.14} & \se{0.34} & \se{0.13} \\ 
    \multirow{2}{*}{Distance Squared}                   & 2.30**  & 1.79**  & 1.47**  & 2.29** \\& \se{0.12} & \se{0.12} & \se{0.26} & \se{0.12} \\ 
    \multirow{2}{*}{In the Same Neighborhood}           & 0.24**  & 0.22**  & 0.24**  & 0.24** \\& \se{0.03} & \se{0.03} & \se{0.04} & \se{0.03} \\ 
    \multirow{2}{*}{Distance * Same Ward}               &         &         & -0.87*  &        \\&           &           & \se{0.36} &           \\ 
    \multirow{2}{*}{Distance Squared * Same Ward}       &         &         & 0.57*   &        \\&           &           & \se{0.29} &           \\ 
    \multirow{2}{*}{Same Neighborhood * Same Ward}      &         &         & -0.19** &        \\&           &           & \se{0.06} &           \\ 
    \multirow{2}{*}{Absolute Difference in Square Feet} &         &         & -0.15*  &        \\&           &           & \se{0.07} &           \\ 
    \multirow{2}{*}{Absolute Difference in Bedrooms}    &         &         & -0.55** &        \\&           &           & \se{0.08} &           \\ 
    \multirow{2}{*}{Absolute Difference in List Price}  &         &         & -0.92** &        \\&           &           & \se{0.09} &           \\ 
    \multirow{2}{*}{Difference in Square Feet * Ward}   &         &         & -0.24   &        \\&           &           & \se{0.13} &           \\ 
    \multirow{2}{*}{Difference in Bedrooms * Ward}      &         &         & -0.33** &        \\&           &           & \se{0.09} &           \\ 
    \multirow{2}{*}{Difference in List Price * Ward}    &         &         & -1.40** &        \\&           &           & \se{0.12} &           \\ 
    \multirow{2}{*}{Very High Price}                    &         &         &         & 0.21** \\&           &           &           & \se{0.07} \\ 
    \multirow{2}{*}{Very Low Price}                     &         &         &         & 0.14   \\&           &           &           & \se{0.08} \\ 
    \multirow{2}{*}{Very High Price * Ward}             &         &         &         & 0.19** \\&           &           &           & \se{0.07} \\ 
    \multirow{2}{*}{Very Low Price * Ward}              &         &         &         & -0.42**\\&           &           &           & \se{0.08} \\ 
    \multirow{2}{*}{Constant}                           & -3.02** & -3.04** & -3.04** & -3.04**\\& \se{0.05} & \se{0.05} & \se{0.12} & \se{0.05} \\ 
    \bottomrule
  \end{tabular}
  }
\end{table}
                                                        
\end{document}



